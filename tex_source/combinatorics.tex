\section{Комбинаторика}
Если говорить коротко, то комбинаторика --- это раздел математики,
занимающийся подсчетом. Считать приходится в самых разных задачах,
поэтому данная наука является фундаментальной и необходима к
ознакомлению. Мы рассмотрим самые простые, но крайне важные методы и
понятия.

\subsection{Правила суммы и произведения}

Начнем с двух очень естественных правил: правило суммы и правило
произведения.

\begin{statement}[Правило суммы]
    Если нужно выбрать объект из множества $A$, которое разделено
    на непересекающиеся множества $B$ и $C$, то сделать это можно
    $|B| + |C|$ способами.
\end{statement}

\begin{statement}[Правило произведения]
    Если объект строится в два этапа, причем на первом этапе
    производится выбор из $m$ вариантов, а на втором этапе --- из $n$, 
    то всего различных объектов ровно $mn$.
\end{statement}

Посмотрим, как работают эти правила на двух простых примерах.

\begin{task}
    Пусть имеются 4 города: $A, B, C, D$. Из $A$ в $B$ ведут две
    дороги. Из $B$ в $D$ --- три, из $A$ в $C$ есть единственный
    путь, а из $C$ в $D$ можно добраться четырьмя способами. Все
    дороги односторонние, других дорог нет. Сколько различных путей
    есть из города $A$ в город $D$?
\end{task}
Можно разделить все пути на два множества: множество $M_B$ путей,
проходящих через город $B$, и $M_C$ --- пути, проходящие через город
$C$. Тогда число всех путей есть $|M_B| + |M_C|$. Каждый путь из
$A$ в $D$ состоит из двух частей, причем выбираются части независимо.
Значит, число путей из $A$ в $D$ равно $2 \cdot 3 + 1 \cdot 4 = 10$.

\begin{task}
    Сколько существует пятизначных чисел, в которых есть хотя бы
    одна четная цифра (число может начинаться с нуля)?
\end{task}
Обозначим количество интересующих нас чисел за $C$. Посчитать явно
такие числа сложно, но можно поступить иначе: посчитать количество
$A$ всех пятизначных чисел, после чего вычесть количество $B$
пятизначных чисел, в которых все цифры нечетные. Действительно,
поскольку мы разделили все числа на два множества, то $A = B + C$,
откуда $C = A - B$.

Осталось найти $A$ и $B$. Пятизначное число строится следующим
образом: пять раз для каждого разряда выбирается произвольная цифра
от 1 до 10. Значит, таких чисел $10^5$. Аналогично получаем, что
чисел, состоящих из нечетных чисел, ровно $5^5$. Итого, $C = 10^5
-5^5$.

\subsection{Урновые схемы}

Попробуем теперь обобщить наши результаты. Для начала вспомним, что
такое факториал числа: $n! = n\cdot (n-1) \cdot \ldots \cdot 2 \cdot 
1$. Пусть у нас имеется урна с $n$ различными шарами. Мы хотим выбрать 
$k < n$ шаров. Сколькими способами это можно сделать. Вообще говоря,
возможны 4 ситуации в зависимости о того, какие выборы мы считаем
различными и как организован выбор.
\begin{enumerate}
    \item Мы учитываем порядок, шары можно возвращать обратно. В этом
        случае все просто. Мы каждый раз выбираем независимо один из
        $n$ шаров, получая различные выборки. Всего $n^k$ вариантов.
    \item Мы снова учитываем порядок, но шары возвращать нельзя.
        Тогда первый шар можно выбрать $n$ способами, следующий ---
        $(n-1)$ и так далее. Всего
        \[
            n(n-1)\ldots (n-k+1) = \frac{n!}{(n-k)!}
            \]
        вариантов. Аналогичными рассуждениями получаем, что $n!$
        есть число способов переставить $n$ различных элементов.
    \item Мы не учитываем порядок, шары возвращать нельзя. В этом
        случае ответ похож на предыдущий. Однако в предыдущей схеме
        мы много раз посчитали одни и те же варианты, а именно: мы
        каждый вариант посчитали столько раз, сколькими способами
        можно переставить $k$ шаров, то есть $k!$. Получается, в
        данном случае вариантов
        \[
            \binom{n}{k} = \frac{n!}{(n-k)!k!}.
            \]
    \item Порядок не учитывается, шары можно возвращать. Эта задача
        немного сложнее, она сводится к задаче о шарах и перегородках.
        Ее мы предлагаем решить читателю самостоятельно.
\end{enumerate}

\subsection{Биномиальные коэффициенты}

Число $\binom{n}{k}$, которое мы получили в одной из урновых схем,
называется биномиальным коэффициентом или числом сочетаний из $n$
элементов по $k$ элементов. Оно равно числу способов из $n$-элементного
множества выбрать подмножество, состоящее из $k$ элементов.

С этими числами связано такое равенство, как бином Ньютона.

\begin{theorem}[Бином Ньютона]
    Для произвольных переменных $a,\ b$ и натурального $n$ верно
    \[
        (a+b)^n = \Sum_{k=0}^n \binom{n}{k} a^k b^{n-k}.
        \]
\end{theorem}
\begin{proof}
    Достаточно раскрыть скобки в выражении
    \[
        \underbrace{(a+b) \ldots (a+b)}_{n\ \text{раз}}
        \]
    и посчитать количество слагаемых вида $a^kb^{n-k}$. Но ясно, что
    оно равно числу сочетаний из $n$ по $k$, поскольку нам достаточно
    выбрать из $n$ скобок ровно $k$, из которых мы возьмем множитель
    $a$. Из остальных скобок возьмем множитель $b$ и получим
    слагаемое данного вида.
\end{proof}

\begin{corollary}
    Для любого натурального $n$ выполнено
    \[
        \binom{n}{0} + \binom{n}{1} + \ldots + \binom{n}{n} = 2^n.
        \]
\end{corollary}
\begin{proof}
    Это полезное равенство можно доказывать многими способами, но
    оно сразу следует из бинома Ньютона подстановкой $a = b = 1$.
\end{proof}

Еще одним важным свойством биномиальных коэффициентов является
рекуррентная формула.

\begin{statement}
    Для натуральных $k, n$ таких, что $k \leqslant n$, выполнено
    \[
        \binom{n}{k} = \binom{n-1}{k-1} + \binom{n-1}{k}.
        \]
\end{statement}
\begin{proof}
    Доказывать это равенство также можно различными способами. Мы
    докажем алгебраически:
    \begin{multline*}
        \binom{n-1}{k-1} + \binom{n-1}{k} = \frac{(n-1)!}{(k-1)!
        (n-k)!} + \frac{(n-1)!}{k!(n-k-1)!} =\\\\= \frac{n!}{k!(n-k)!}
        \left( \frac{k}{n} + \frac{n-k}{n} \right) = \binom{n}{k}.
    \end{multline*}
\end{proof}
Это свойство лежит в основе треугольника Паскаля. Этот треугольник
выглядит следующим образом: в $n$-й строке (строки пронумерованы 
сверху вниз) длины $n$ на $k$-м месте написано число $\binom{n}{k}$. 
При этом указанное свойство можно переформулировать так: каждое число 
в треугольнике Паскаля (кроме самого <<верхнего>>) равно сумме двух 
стоящих над ним чисел.
