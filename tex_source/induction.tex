\section{Индукция}
\subsection{Что такое индукция?}
Принцип математической индукции является очень важным и широко
применяется во многих областях как математики, так и других наук.
Прежде чем формализовать его, рассмотрим два классических примера.

\begin{task}
    На плоскости проведено несколько прямых. Они делят плоскость
    на области. Докажите, что области можно так раскрасить в два
    цвета, что соседние области покрашены в разные цвета.
\end{task}
Смысл рассуждения по индукции заключается в том, чтобы от более
простых конструкций переходить к более сложным. Для начала посмотрим,
что будет, если прямая одна. В этом случае мы просто раскрашиваем
полуплоскости в разные цвета. Если прямых хотя бы две, то сразу
появляются варианты --- некоторые из прямых могут быть параллельны.
А что если прямых еще больше? 

Теперь посмотрим, что происходит, когда мы добавляем ровно одну
прямую. Предположим, имеется конфигурация из $n$ прямых, которая
правильно раскрашена в два цвета. Добавим новую прямую $l$. Она
пересечет остальные не более чем в $n$ точках. Значит, она будет
содержаться максимум в $n+1$ области. Назовем такие области
пограничными и посмотрим на нашу раскраску.

Каждая пограничная область разбилась на две части, окрашенные в один
цвет, причем части эти лежат по разную сторону от прямой $l$. Тогда
перекрасим каждую область с одной стороны от $l$ в противоположный
цвет. Теперь пограничные области раскрашены правильно, а все
остальные соседние области либо не поменяли цвет, либо поменяли цвет
одновременно. Поскольку исходная раскраска была правильной, данная
раскраска также удовлетворяет требованиям задачи.

Таким образом, мы привели правильную раскраску для одной прямой и
объяснили, как получить новую раскраску при добавлении произвольной
прямой. Значит, мы можем покрасить плоскость для любого расположения
прямых.

\begin{task}[Ханойские башни]
    Имеются три стержня. На одном из них находится пирамида из колец
    с уменьшающимся снизу вверх размером. Нельзя перекладывать
    несколько колец и нельзя класть большее кольцо на меньшее.
    Можно ли произвольное количество колец переложить с одного стержня
    на другой?
\end{task}

Снова попробуем начать с небольшого количества --- одно кольцо мы
всяко умеем перекладывать. Допустим, мы научились решать задачу для
$n$ колец, а нам нужно перетащить $n+1$ кольцо на, скажем, второй
стержень.

Решение получится очень простым. Мы перенесем верхние $n$ колец (так
как нижнее кольцо самое большое, можно его не трогать и считать, что
его нет) на третий стержень. Далее, самое большое кольцо переместим
на второй стержень и вернем $n$ колец с третьего стержня на второй.

Теперь мы можем формально зафиксировать изложенный метод:

\begin{statement}[Принцип математической индукции] Пусть имеется 
    некоторое утверждение $A(n)$, зависящее от натурального параметра 
    $n$. Если
    \begin{enumerate}
        \item Утверждение верно при $n=1$,
        \item Для любого $k \in \N$ из <<$A(k)$ верно>> следует, что
            <<$A(k+1)$ верно>>,
    \end{enumerate}
    то утверждение $A(n)$ верно для всех натуральных $n$.
\end{statement}

Первый пункт называется базой индукции, а второй --- переходом (или
шагом). В первой задаче базой была явная раскраска плоскости в случае
$n=1$, а переходом --- добавление новой прямой. В задаче о Ханойских
башнях база --- это решение для одного кольца, а переход --- решение
для $n+1$ кольца с использованием решения для $n$ колец.

Попробуем теперь применить этот метод к алгебраическим задачам.

\subsection{Формулы суммирования}
Рассмотрим простую и всем известную сумму:
\begin{task}
    Докажите, что
    \[
        1 + 2 + \ldots + n = \Sum_{k=1}^n k = \frac{n(n+1)}2.
        \]
\end{task}
Эта задача решается просто, даже если мы не знаем ответа заранее.
Достаточно обозначить сумму за $S$ и сложить ее с собой, записанной
в обратном порядке, то есть
\[
    \begin{cases}
        1 + 2 + \ldots + n = S\\
        n + (n-1) + \ldots + 1 = S
    \end{cases}
    \quad \Longrightarrow \quad
    \underbrace{(n+1) + (n+1) + \ldots + (n+1)}_{n\ \text{раз}} = 2S,
    \]
откуда получаем выражение для $S$. Однако, если ответ известен,
можно рассуждать по индукции. Действительно, при $n = 1$ равенство
тривиально:
\[
    1 = \frac{1(1+1)}2.
    \]
Пусть теперь оно выполнено для некоторого $k$. Докажем, что в этом
случае оно выполнено и для $k+1$. В самом деле,
\[
    \underbrace{1 + 2 + \ldots + k}_{\text{пользуемся предположением}}
    + (k+1) = \frac{k(k+1)}2 + (k+1) = \frac{(k+1)(k+2)}2.
    \]
В данной задаче можно было обойтись и без индукции. А, к примеру,
формулу
\[
    1 + 4 + \ldots + n^2 = \Sum_{k=1}^n k^2 = \frac{n(n+1)(2n+1)}6
    \]
так просто доказать не получится. А по индукции рассуждение почти
не отличается от предыдущего.

С помощью метода математической индукции можно доказывать не только
равенства, но и неравенства.

\subsection{Неравенства}
Начнем с классического и очень полезного результата:
\begin{theorem}[Неравенство Бернулли]
    При всех натуральных $n$ и $h \geqslant -1$ верно неравенство
    $(1+h)^n \geqslant 1 + hn$.
\end{theorem}
\begin{proof}
    Воспользуемся индукцией по $n$. База индукции тривиальна: $1 + h
    \geqslant 1 + h$. Докажем переход. Пусть $(1+h)^k \geqslant 1 + 
    hk$. Тогда
    \[
        (1+h)^{k+1} = (1+h)(1+h)^k \geqslant (1+h)(1+hk) = 
        1 + h(k+1) + h^2k \geqslant 1 + h(k+1).
        \]
\end{proof}

Однако не всегда получается сразу воспользоваться рассуждением по
индукции. Иногда приходится сделать усиление утверждения.
\begin{task}
    Докажите, что для $n \in \N$ выполнено
    \[
        1 + \frac1{2^2} + \ldots + \frac1{n^2} < 2.
        \]
\end{task}
Если мы сразу попробуем применить индукцию, у нас возникнут проблемы
с переходом. Но можно усилить задачу следующим образом: доказывать
неравенство
\[
    \Sum_{k=1}^n \frac1{k^2} \leqslant 2 - \frac1n.
    \]
Снова проверяем базу индукции: $1 \leqslant 2 - 1$. Теперь докажем
переход от $k$ к $k+1$:
\begin{multline*}
    1 + \frac14 + \frac19 + \ldots + \frac1{k^2} + \frac1{(k+1)^2} 
    \leqslant 2 - \frac1k + \frac1{(k+1)^2} =\\\\=2 - \frac{k^2 + k + 
    1}{k(k+1)^2} < 2 - \frac{k^2 + k}{k(k+1)^2} = 2 - \frac1{k+1}.
\end{multline*}

Несмотря на то, что индукция кажется довольно понятным инструментом,
даже в простых рассуждениях можно допустить ошибки.

\subsection{Ошибки в рассуждениях}
Рассмотрим пару несложных примеров.
\begin{example}
    Докажем, что соседние натуральные числа равны. Действительно, 
    пусть для первых $n$ пар чисел это верно. В частности, $n = n + 1$.
    Прибавив единицу, получаем, что $n+1 = n+2$. 
\end{example}
В данном случае ошибка совсем проста, но поучительна. Мы не проверили
базу, то есть что $1 = 2$. Часто проверка базы индукции существенно
легче, чем доказательство перехода, однако забывать про нее нельзя.

\begin{example}
    Докажем, что любые $n$ чисел $a_1, \ldots, a_n$ равны. Проверим
    базу. При $n = 1$ число всего одно, и утверждение верно. Если
    чисел $n$, то уберем по очереди $a_1$ и $a_n$ и, применив
    предположение индукции, получим
    \[
        a_1 = a_2 = \ldots = a_{n-1} \qquad \text{и} \qquad
        a_2 = \ldots = a_{n-1} = a_n.
        \]
    Объединив два равенства, получаем $a_1 = a_2 = \ldots a_n$.
\end{example}
Здесь ошибка уже не так очевидна. Наше рассуждение в доказательстве
перехода ломается при $n=2$. В этом случае при вычеркивании одного
числа остается также одно число, и равенств выписать не удается.
