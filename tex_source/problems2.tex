\section{Задачи}
\subsection{Упражнения}
\begin{enumerate}
    \item Докажите, что не существует графа с пятью вершинами, степени
        которых равны 4, 4, 4, 4, 2.
    \item В графе 100 вершин и 800 ребер. Докажите, что в этом графе
        есть хотя бы одна вершина степени не меньше 16. Может ли у
        каждой вершины быть степень больше 16?
    \item В связном графе $n$ вершин. Сколько в нем может быть ребер?
        Найдите наибольшее и наименьшее число.
    \item В графе $2n$ вершин, степень каждой не менее $n$. Обязательно
        ли в этом графе есть цикл длины 3? Длины 4?
    \item В графе из каждой вершины выходит ровно 3 ребра. Может ли в
        таком графе быть 2017 ребер?
    \item Докажите, что не существует графа, степени вершин которого
        попарно различны.
    \item В некоторой стране из каждого города выходит ровно 64 дороги
        в другие города. Из каждого города можно добраться до любого
        другого. Докажите, что при закрытии любой дороги из каждого
        города все еще можно будет добраться до любого другого.
    \item Докажите, что среди 6 человек найдутся либо 3 попарно
        знакомых, либо 3 попарно незнакомых человека.
    \item Дерево на $n$ вершинах --- это связный граф на $n$ вершинах,
        в котором нет циклов. Докажите, что в таком графе есть
        вершина степени 1 (висячая вершина) при $n \geqslant 2$.
        Докажите, что висячих вершин хотя бы 2.
    \item Докажите, что из связного графа можно выкинуть несколько
        ребер так, чтобы осталось дерево (такое дерево называется
        остовным).
    \item Докажите, что в дереве на $n$ вершинах ровно $n-1$ ребро.
    \item Докажите, что в связном графе на $n$ вершинах хотя бы
        $n-1$ ребро.
    \item Докажите, что если в связном графе на $n$ вершинах ровно
        $n-1$ ребро, то это дерево.
    \item Сколько существует правильных раскрасок графа-пути длины
        $n$ (с $n+1$ вершиной) в три цвета?
    \item Сколько существует правильных раскрасок дерева на $n$
        вершинах в два цвета?
    \item В некоторой стране каждый город соединен с каждым дорогой
        с односторонним движением. Докажите, что найдется город, из
        которого можно добраться в любой другой.
    \item Куб состоит из $n^3$ единичных кубиков. Какое минимальное
        число перегородок между единичными кубиками нужно удалить,
        чтобы из каждого кубика можно было добраться до границы куба?
\end{enumerate}

\subsection{Задачи с олимпиад}
\begin{enumerate}
    \item В один из дней года оказалось, что каждый житель города 
        сделал не более одного звонка по телефону. Докажите, что
        население города можно разбить не более, чем на три группы 
        так, чтобы жители, входящие в одну группу, не разговаривали в 
        этот день между собой по телефону.
        \begin{hint}
            Докажите, что есть житель, который разговаривал не более
            чем с двумя другими людьми, и воспользуйтесь индукцией 
            по числу жителей.
        \end{hint}
    \item В некотором городе на любом перекрестке сходятся ровно 3
        улицы. Улицы раскрашены в три цвета так, что на каждом
        перекрестке сходятся улицы трех разных цветов. Из города
        выходят три дороги. Докажите, что они имеют разные цвета.
        \begin{hint}
            Посчитайте улицы каждого цвета (выразите их число через
            количество вершин и выходящих дорог каждого цвета).
            Воспользуйтесь равенством для числа выходящих дорог
            и четностью.
        \end{hint}
    \item В стране несколько городов, некоторые пары городов
        соединены беспосадочными рейсами одной из $N$ авиакомпаний,
        причем из каждого города есть ровно по одному рейсу каждой из
        авиакомпаний. Известно, что из любого города можно долететь
        до любого другого (возможно, с пересадками). Из-за
        финансового кризиса был закрыт $N-1$ рейс, но ни в одной
        из авиакомпаний не закрыли более одного рейса. Докажите,
        что по-прежнему из любого города можно долететь до любого
        другого.
        \begin{hint}
            Оставьте ребра только двух авиакомпаний, одна из которых
            вышла из кризиса без потерь. Посмотрите, какой граф
            получился.
        \end{hint}
    \item В компании из $2n+1$ человек для любых $n$ человек
        найдется отличный от них человек, знакомый с каждым из них.
        Докажите, что в этой компании есть человек, знающий всех.
        \begin{hint}
            Попробуйте постепенно построить достаточно большую клику
            (подмножество вершин, между которыми проведены все
            возможные ребра) и посмотрите на оставшиеся вершины.
        \end{hint}
    \item В стране $N$ городов. Между любыми двумя из их проложена
        либо автомобильная, либо железная дорога. Турист хочет
        объехать страну, побывав в каждом городе ровно один раз,
        и вернуться в город, с которого он начинал путешествие.
        Докажите, что турист может выбрать город, с которого он
        начнет путешествие, и маршрут так, что ему придется поменять
        вид транспорта не более одного раза.
        \begin{hint}
            Выберите произвольную вершину и воспользуйтесь индукцией
            по $N$.
        \end{hint}
    \item В кабинете стоят 2017 телефонов, любые два из которых
        соединены проводом одного из четырех цветов. Известно, что
        провода всех четырех цветов присутствуют. Всегда ли можно
        выбрать несколько телефонов так, чтобы среди соединяющих
        их проводов встречались провода ровно трех цветов?
        \begin{hint}
            Воспользуйтесь индукцией по числу телефонов. Попробуйте
            выбрать для этого произвольный телефон и разберитесь с
            <<плохим>> случаем.
        \end{hint}
    \item В стране несколько городов, некоторые пары городов 
        соединены двусторонними беспосадочными авиалиниями,
        принадлежащими $k$ авиакомпаниям. Известно, что любые две
        линии однной авиакомпании имеют общий конец. Докажите, что 
        все города можно разбить на $k + 2$ группы так, что никакие 
        два города из одной группы не соединены авиалинией.
        \begin{hint}
            Посмотрите, как могут выглядеть графы авиалиний для
            авиакомпаний воспользуйтесь индукцией по $k$. Для
            случая, когда подграфы суть треугольники воспользуйтесь
            рассуждением от противного и оцените число ребер двумя
            способами.
        \end{hint}
\end{enumerate}
