\section{Задачи}

\begin{enumerate}
    \item Докажите, что в задаче о Ханойских башнях для перекладывания
        $n$ колец достаточно $2^n - 1$ ходов.
    \item Докажите, что для всякого $n \in \N$ квадрат $2^n \times
        2^n$, у которого вырезан угловой квадратик $1 \times 1$, можно
        разрезать на уголки из трех клеток. Тот же вопрос, если
        вырезан произвольный квадратик $1 \times 1$.
    \item Докажите, что при всех натуральных $n$
        \[
            \Sum_{k=1}^n k^2 = \frac{n(n+1)(2n+1)}6.
            \]
    \item Докажите, что для всех $n \in \N$ сумма
        \[
            1^3 + 2^3 + \ldots + n^3
            \]
        является точным квадратом.
        \begin{hint}
            Усильте утверждение, найдя точное значение суммы.
        \end{hint}
    \item Докажите для произвольного $n \in \N$ неравенство
        \[
            1 + \frac12 + \frac14 + \ldots + \frac1{2^n} < 2.
            \]
        \begin{hint}
            Замените неравенство на более сильное равенство.
        \end{hint}
    \item Докажите для натурального $n$ равенство
        \[
            1 + 2 + 4 + \ldots + 2^n = 2^{n+1} - 1.
            \]
    \item Докажите для натурального $n$ равенство
        \[
            1 + 3 + 5 + \ldots + (2n - 1) = n^2.
            \]
    \item На доске написаны сто цифр --- нули и единицы (в любой
        комбинации). Разрешается выполнять два действия: заменять
        первую цифр на другую и заменять цифру, стоящую после первой
        единицы. Докажите, что после нескольких таких замен можно
        получить любую комбинацию из 100 нулей и единиц.

    \item Сколько чисел от 1 до 1000 делятся на 2? На 3? На 6?

    \item Сколько существует четырехзначных чисел, в которые
        входит цифра 7 (число может начинаться с нулей)?

    \item Сколько существует пятизначных чисел (число не может
        начинаться с нулей)? Все цифры которых четные?

    \item Сколько существует слов, составленных из букв слова
        ГРАФ? Слова ИНДУКЦИЯ? Слова МАТЕМАТИКА?

    \item Дана шахматная доска. Король стоит в клетке $a1$ и может
        ходить только вверх и вправо. Сколькими способами он может
        добраться до клетки $f4$?

    \item Покажите двумя способами, что при натуральных $k < n$.
        \[
            \binom{n}{k} = \binom{n}{n-k}.
            \]

    \item Докажите комбинаторно, что для всякого натурального $n$
        \[
            \binom{n}{0} + \binom{n}{1} + \ldots + \binom{n}{n} = 2^n.
            \]

    \item Докажите комбинаторно рекуррентную формулу для 
        биномиальных коэффициентов.
    
    \item Сколько диагоналей у выпуклого $n$-угольника?

    \item Есть 10 различных предметов. 4 человека хотя унести по
        одному предмету каждый. Сколькими способами они это могут
        сделать?

    \item 10 человек хотят встать в очередь. Сколькими способами они
        могут это сделать? Тот же вопрос про хоровод.

    \item Докажите, что для натурального $n$ выполнено
        \[
            \binom{n}{0} - \binom{n}{1} + \binom{n}{2} - \ldots 
            + (-1)^n \binom{n}{n} = 0.
            \]
        Что это значит с точки зрения комбинаторики?

    \item Для $n \in \N$ найдите сумму $\sum_k \binom{n}{k}$, где
        суммирование ведется по нечетным $k < n$. По четным.

    \item Сколькими способами можно выписать в ряд цифры от 0 до 9
        так, чтобы четные цифры шли в порядке возрастания, а
        нечетные --- в порядке убывания?

    \item Докажите для натурального $n$ комбинаторно равенство
        \[
            \Sum_{k=1}^n k\binom{n}{k} = n2^{n-1}.
            \]
        \begin{hint}
            Попробуйте выбрать из $n$ человек команду произвольного
            размера и капитана в ней.
        \end{hint}

    \item Задача о шарах и перегородках. Пусть есть $k$ разных ящиков 
        и $n$ одинаковых шаров. Сколькими способами можно разложить 
        эти шары по ящикам?
        \begin{hint}
            Попробуйте выбрать места, куда поставить перегородки,
            отвечающие границам ящиков.
        \end{hint}

    \item Сведите урновую схему с возвращением и без учета порядка
        к задаче о шарах и перегородках и найдите ответ.
\end{enumerate}
